\documentclass{article}                 %使用article格式
 \topmargin=0pt                          %以下页面设置,\topmargin页眉到页边的距离
 \oddsidemargin=0pt \evensidemargin=0pt % 奇数页的左面页边距(奇数页(右页)中,正文的左端到左基准线的距离);偶数页的左面页边距
 \textwidth=14cm                         %宽度
 \textheight=21cm                        %高度
                                         %
 \title{This is a sample of \LaTeX}      %文章题目,\latex专有 
 \author{Bigeyes(\tt{chencs@126.net})\\ %作者. 多行用\\
         Department of XXXX\\     %
         UESTC, Chengdu, Sichuan, 611731      %删除下一个作者,添加}
                                         
 \and                                  %如有多作者, 用\and
         A\_A                            %
 \thanks{Supported by ...}             %致谢, 在author里
         \\                              %
         XXX of UESTC\\                   %
         wwww.uestc.edu.cn}   %
 \date{Mar. 1, 2017}                     %日期, 如果没有此项,
                                         %则以当前日期代替, 若
                                         %不要日期, 则用空{}
                                         %
 \begin{document}                        %document开始.
                                         %以上定义标题各项内容,
 \maketitle                              %此句产生标题, 不可缺少,
                                         
 \begin{abstract}                        %套用abstract格式
 This is mini-page defined for abstract, %自动形成小页
 you only write your abstract in it. If %
 you want to shows keywords, maybe you   %
 should use:                             %

 {\bf Keywords: }\LaTeX, example         %LaTeX没有keywords环境
 \end{abstract}                          
                                         %
 \section{The very beginning}            %一个节,
 This is the first section of your       %
 article. You may find every first       %一节的第一段缺省情况
 paragraph of your section, subsection, %为段首不自动缩进
 chapter or ... always has no            %
 ``parindent'' at the beginning.         %

 This is the second paragrph, you can    %但是以后的段段首
 find this has parindent at the          %         自动缩进
 beginning. If you want to show          %
 parindent at first paragraph too,       %
 do as the first paragraph I showed      %
 in the next section.                    %
                                         %
 \section{The 2nd step}                  %第二节,使用自动标号
 \hskip \parindent                       %横向空出\parindent
 This is the second section. In this     % 
 first paragraph, I use `hskip' to       % 
 get the first parindent. Maybe you      % 
 can get this effect by another way.     % 
                                         %
 \subsection{Sub-sect of 2}              %小节
 this                                    %
                                         %
 \subsection*{\S 2.2 Another sub of 2}   %小节可以不使用
 this                                    %自动标号, 这时有关
                                         %的计数器不增加
 \section{Conclusion}                    %
 I think you have know \TeX well now.    %又是一节
 I want to show you how to use           %
 bibliography. In the article, you       %参考文献的
 can use as ``see \cite{texbook}''.      %引用方法\cite{标识}
                                         %
 \begin{thebibliography}{0}              %参考文献列在这里
                                         %{0}表示最长文献序号
                                         %           为一位数
                                         %\bibitem{标识}
                                         %参考文献使用后详
 \bibitem{texbook} Donald~E.~Knouth, ``The \TeX book'',
 Addison-Wesley, 1984
 \bibitem{lamport} L.\ Lamport, ``\LaTeX:
         A Document Preparation System'',
         Addison-Wesley, 1994
 \bibitem{companion} M.~Goossens, F.~Millelbach,
         and A.~Samarin, ``The \LaTeX\ Companion'',
         Addison--Wesley, 1994
 \end{thebibliography}

 \end{document}                          %document结束