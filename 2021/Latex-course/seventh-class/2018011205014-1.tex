\documentclass{beamer}
\usepackage[UTF8]{ctex}
\usepackage{ulem}
\usepackage{amsmath}
\usepackage{amssymb}
\usepackage{geometry}
\usepackage{graphicx}  
\usepackage{subfigure}  
\usepackage{fancyhdr}
\usepackage{listings}
\usepackage{color}
\usepackage{pfnote}
\usepackage{listings}
\usepackage{xcolor}
\usepackage{cite}
\usetheme{Warsaw}
\begin{document}
\title{Sliders' title 标题}%这两个就不能加\了 
\author{Authors 作者} 
\date{\today}%today 前面也要补\。
\begin{frame}
\titlepage % Print the title page as the first slide
\end{frame}
\begin{frame}
\frametitle{Overview} % Table of contents slide, comment this block out to remove it
\tableofcontents % Throughout your presentation, if you choose to use \section{} and \subsection{} commands, 
these will automatically be printed on this slide as an overview of your presentation
\end{frame}
\begin{frame}
\frametitle{幻灯片测试} %\pause
我的第一张幻灯片。
\begin{theorem}[Mass--energy equivalence]
$E = mc^2$
\end{theorem}
\begin{definition}
definition 1...%原来这个不能加\的。。
\end{definition}
\end{frame}
\begin{frame}
 \frametitle{第二张幻灯片}
 \pause
 \begin{itemize}
 \item beamer introduction pause
 \item beamer details pause
 \item beamer conclusions
 \end{itemize}
 \end{frame}
\end{document}