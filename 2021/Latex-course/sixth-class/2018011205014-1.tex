\documentclass[11pt]{article}
\usepackage{fullpage}
\usepackage{cTex}
\usepackage{amssymb}
\usepackage{amsmath}
\usepackage{geometry}
\usepackage{threeparttable}
\usepackage{indentfirst} 
\usepackage{xeCJK}
\let\kaishu\relax
\newCJKfontfamily\kaishu{KaiTi}[AutoFakeBold]
\setlength{\parindent}{2em}
\textheight=9.0in
\pagestyle{empty}
\raggedbottom
\raggedright
\setlength{\tabcolsep}{0in}
\begin{document}
{\kaishu {
    \begin{center}
        {\LARGE \textbf{杨睿妮}}\\
        13281835536\\
        yangruinii@foxmail.com\\
        四川省成都市西源大道2006号
    \end{center}
\bigskip
{\Large \textbf{\underline{教育背景}}}\\
\begin{itemize}
    \item {\textbf{学士,电子科技大学}} \hfill{2018.09 - 至今} \\
    英才实验学院(工科实验班),计算机科学专业\\
    \item GPA: 3.99 / 4.00,CET-4 成绩 601,CET-6 成绩 553 \\
    \item \textbf{数理课程成绩:}数学分析 87、随机数学及概率论 93、线性代数 91、离散数学 98\\ 
    \textbf{计算机专业课成绩:}数据结构与算法 85、操作系统 91、计算机网络 87、人工智能 99
\end{itemize}

\bigskip
{\Large \textbf{\underline{项目\&研究经历}}}\\
\begin{itemize}
    \item {\textbf{基于强化学习的非真实感草稿渲染简化}} \hfill{2020.09 - 至今} 
    
    \qquad 通过编码器对图像降维,利用了强化学习框架,将艺术家的草稿进行简化,以生成直接用于上色的线稿。研究相关领域已有成果,参考了早稻田大学的一系列线稿简化工作,基于其结果上优化。\\
    \item {\textbf{基于 Flutter 的聊天应用}} \hfill{2020.12} 
    

    \qquad 属于个人项目,尝试使用谷歌的 Flutter 跨平台应用开发框架实现类似 QQ 简洁版的界面 UI,并计划使用 Leancloud 的即时聊天 API 以及声网的音视频 API 完成完整的聊天功能。

    \item {\textbf{人工智能课程设计}} \hfill{2020.09 - 2020.11} 
    

    \qquad 主要包括 A* 启发式搜索算法解决八数码问题、手动实现决策树的建立和剪枝过程、手动实现包含一个隐层和 Sigmoid 激活函数的神经网络的反向传播算法。使用 Python 实现,严格遵守 PEP8 代码风格规范,使用 Python 3 Typing 系统提升代码的可阅读性。


\end{itemize}

\bigskip
{\Large \textbf{\underline{荣誉\&奖项}}}\\
\begin{itemize}
    \item {\textbf{Google HashCode 编程挑战赛,国际Top 15\%}} \hfill{2021.02} \\
    \item {\textbf{APMCM 亚太地区大学生数学建模二等奖}} \hfill{2020.11}


\end{itemize}

}}

{}


\end{document}