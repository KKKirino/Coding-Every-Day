\documentclass[11pt]{article}
\usepackage{ctex}
\usepackage{fontspec}
\usepackage{tabularx}
\usepackage{fourier}
\usepackage{Hologo}
\usepackage{geometry}
\usepackage[T1]{fontenc}
\usepackage[utf8]{inputenc}
\usepackage{authblk} 
\usepackage{color, xcolor}       
\usepackage{ulem}
\usepackage{soul}
\geometry{a4paper,left=2cm,right=2cm,top=1cm,bottom=2cm}
\title{这是我的第一份 \LaTeX\ 文档}
\songti
\setmainfont{Times New Roman}  
\author{ 杨睿妮 ( 邮 箱 :1416333092@qq.com)\\ Department of Information and Communication
Technology\\UESTC,Chengdu,Sichuan,611731}
\date{March 10, 2021}
\XeTeXlinebreaklocale "zh"
\XeTeXlinebreakskip = 0pt plus 1pt
\begin{document}
\begin{sloppypar}
\maketitle
\section{What is a novel coronavirus?}   
\begin{flushright}
    \textbf{A novel coronavirus is a new coronavirus that has not been previously identified.} \newline
    The virus causing coronavirus disease 2019 (COVID-19), is not the same as the coronaviruses that commonly circulate among humans and cause mild illness, like the common cold.

\end{flushright}   
\subsection{Introduction}
\textbf{\emph{\ul{A diagnosis with coronavirus 229E, NL63, OC43, or HKU1 is not the same as a COVID-19 diagnosis. Patients with COVID-19 will be evaluated and cared for differently than patients with common coronavirus diagnosis.
.}}}
    \subsubsection{最近新闻}
    \begin{center}
    \uline{针对新冠疫情与秋冬季呼吸道传染病叠加的风险,9月15日至10月15日,市爱卫会、市健康促进委在全市开展“秋冬季防病科普宣传月”活动。}\\
    \uuline{将通过“百万市民防疫知识与健康素养大赛”“健康上海说”“健康公开课”“健康大讲堂”等活动,向广大市民普及秋冬季防病知识和《上海市民健康公约》等内容,持续提升2400万市民自我防护意识和能力,守牢城市公共卫生安全底线。新冠肺炎防控进入常态化阶段,本市疫情防控向好态势进一步巩固。但与此同时,境外疫情持续扩散蔓延,病例输入性风险仍然较大。秋冬季是常见呼吸道传染病高发季节,存在新冠肺炎疫情与呼吸道传染病流行叠加的风险。上海市健康促进中心为广大市民及时速递健康提示:做到这“五要”,把疾病赶跑。}
    \end{center}
    \newpage
\section{防护措施} 
\subsection{口罩还是要戴}
\begin{center}
    提倡随身携带口罩,乘坐交通工具、进入医疗机构等有明确要求的场所应佩戴口罩,在密闭空间内、人群密集区或需要与他人密切接触时应及时佩戴,出现感冒、咳嗽症状时自觉佩戴口罩。
\end{center}
    
    \subsection{社交距离要留}  
    \begin{flushright}
        呼吸道传染病流行期间减少前往人群密集处或室内密闭场所,在公共场所内遵守秩序、避免拥挤,自觉与他人保持社交距离,避免与有呼吸道症状如咳嗽、打喷嚏等的人接触。
    \end{flushright} 
    
    \subsection{咳嗽喷嚏要遮} 
    \sout{咳嗽、打喷嚏时尽量避开他人,用纸巾或弯曲的手肘遮挡口鼻,防止飞沫四溅,使用后的纸巾立即丢弃并洗手。}
    \subsection{双手经常要洗} 
    \uline{勤洗手是预防传染病简便有效的措施之一,饭前便后、触摸公共物品后、咳嗽打喷嚏遮掩口鼻后、接触生鲜食材后,都要及时用流动水和肥皂或洗手液,按照七步洗手法规范洗手,尤其牢记不要用脏手触摸眼口鼻。}
    \subsection{窗户尽量要开} 
    经常开窗通风能保持室内空气流通,至少每日通风2次、每次30分钟以上,以形成空气对流为佳。  
\end{sloppypar}
\end{document}
