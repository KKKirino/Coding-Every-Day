\documentclass[12pt]{article}
\usepackage{fontspec}
\usepackage{tabularx}
\usepackage{fourier}
\usepackage{Hologo}
\setmainfont{楷体}
\title{Latex课程第一次作业}
\author{杨睿妮}
\date{}
\XeTeXlinebreaklocale "zh"
\XeTeXlinebreakskip = 0pt plus 1pt
\begin{document}
\begin{flushleft}
    个人信息\par
    我叫杨睿妮,是一名电子科技大学18级的学生.学号为2018011205014。我目前就读于计算机科学与工程学院,计算机科学与技术专业。修读过的核心课程包括数据结构与算法、计算机组成原理、汇编语言以及操作系统等。\par
    \hspace*{\fill} \\
    兴趣爱好\par
    爱好原画绘制和人物立绘设计,经常用Photoshop、SAI以及Procreate进行绘制。热衷于欣赏国内外原画师的作品,观看相关的绘画视频学习他们的技巧。此外,听音乐是我课余时间进行放松的主要方式,经常听的歌曲类型为欧美音乐以及Kpop,并且也喜欢欣赏相关的舞蹈视频等。\par
    \hspace*{\fill} \\
    对\Hologo{LaTeX}的认识\par
    所有格式的获得和调整都是通过coding方式实现,在写作过程中就能对想要的格式进行设计和实现。非常便于输入行间和行内的复杂公式,配合图像识别工具,可以快速输入极其复杂的公式。非常有利于管理图表和公式编号,不像Microsoft Word那样要使用题注和交叉引用才能实现,大大降低了格式调整的时间成本。将参考文献的排版变得非常简单,万一需要在数十上百篇文献中插入一篇新的文献,也丝毫没有引用和排版压力。   
\end{flushleft}
\end{document}
